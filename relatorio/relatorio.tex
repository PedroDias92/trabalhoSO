\documentclass[11pt,twoside,a4paper]{report}
\usepackage[a4paper,left=3cm,right=2cm,top=2.5cm,bottom=2.5cm]{geometry}
\usepackage[colorlinks=true,linkcolor=blue,citecolor=blue]{hyperref}
\usepackage[T1]{fontenc} % to have good hyphenation
\usepackage[utf8]{inputenc} % accented characters in input
\usepackage[portuguese]{babel}
\usepackage{color}
\usepackage{adjustbox}
\usepackage{listings}
\lstset{language=C}
\begin{document}
\title{Trabalho de Sistemas Operativos\\Processamento de um notebook}
\author{
   Sérgio Oliveira~\\
   \texttt{a62134}
   \and
   Pedro Dias~\\
   \texttt{a63389}
}
\date{21 de Maio de 2018}
\maketitle
\raggedbottom
\pagebreak
\pagebreak
\section{Introdução}
Com este relatório iremos demonstrar o desenvolvimento de um programa que faz o processamento de ficheiros de formato \textit{notebook}, formato designado pelo trabalho prático. Ao ser executado, o argumento terá de ter um caminho válido para um ficheiro \textit{notebook} para funcionar corretamente. ~\\
Este relatório está dividido em várias secções que correspondem aos pontos principais do trabalho prático, como visto na página seguinte.

\pagebreak
\tableofcontents


\chapter{Execucão}

\section{Introdução}

A funcionalidade básica do nosso programa está em ler o nosso \textit{notebook} linha a linha e termina a sua execução até que o ficheiro passado como argumento tenha sido lido pelo \textit{while} até à linha final.
\lstinputlisting[breaklines=true, keywordstyle=\color{blue},frame=single,language=C, firstline=179, lastline=179]{notebook.c}

No entanto também pode haver ação humana e, para isso, o sinal SIGINT é enviado para o programa (sinal normalmente relacionado com a combinação de botões Control+C).
Nesse caso, a nossa variável global \textit{running} irá determinar o estado de execução do nosso programa.

\lstinputlisting[breaklines=true, keywordstyle=\color{blue},frame=single,language=C, firstline=35, lastline=38]{notebook.c}

Antes de guardarmos numa variável, a linha lida é processada para que reconheça o conjunto de caratéres especiais:
\begin{itemize}
\item \$; 
\item \$|;
\item \$(número)|;
\item \verb|>>>| e \verb|<<<|
\end{itemize}
Para que sejam reconhecidos para o correto processamento do ficheiro.
Uma linha delimitada por qualquer outra expressão diferente dos itens em cima irá ser ignorada e não interpretada como comando.


\raggedbottom
\pagebreak
\section{Re-processamento}
\section{Detecção de erros / Interrupção}

\chapter{Outras Funcionalidades}
\section{Histórico de comandos anteriores}
\section{Execução de conjuntos de comandos}

\chapter{Conclusão}
\end{document}