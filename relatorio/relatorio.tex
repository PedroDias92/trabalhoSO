\documentclass[11pt,a4paper]{report}
\usepackage[a4paper,left=3cm,right=2cm,top=2.5cm,bottom=2.5cm]{geometry}
\usepackage[colorlinks=true,linkcolor=blue,citecolor=blue]{hyperref}
\usepackage[T1]{fontenc} % to have good hyphenation
\usepackage[utf8]{inputenc} % accented characters in input
\usepackage[portuguese]{babel}
\usepackage{color}
\usepackage{adjustbox}
\usepackage{listings}
\lstset{language=C}
\begin{document}
\title{Trabalho de Sistemas Operativos\\Processamento de um notebook}
\author{
   Sérgio Oliveira~\\
   \texttt{a62134}
   \and
   Pedro Dias~\\
   \texttt{a63389}
}
\date{21 de Maio de 2018}
\maketitle
\raggedbottom
\pagebreak
\pagebreak

% CAP 0 - - - - - - - -  - - - - 

\tableofcontents
\pagebreak
\chapter{Introdução}
Com este relatório iremos demonstrar o desenvolvimento de um programa que faz o processamento de ficheiros de formato \textit{notebook}, formato designado pelo trabalho prático. 

Ao ser executado, o argumento terá de ter um caminho válido para um ficheiro \textit{notebook} para funcionar corretamente.

Relacionado aos temas que falamos, apresentaremos fragmentos de código ao longo do relatório, para que haja melhor compreensão do que estamos a explicar.

Este relatório está dividido em várias secções que correspondem aos pontos mais importantes do trabalho prático, como vemos na página de conteúdos.

\raggedbottom
\pagebreak

% CAP 1 - - - - - - - -  - - - - 

\chapter{Execucão}
\section{Noções básicas}

A funcionalidade básica do nosso programa está em ler o nosso \textit{notebook} linha a linha e termina a sua execução até que o ficheiro passado como argumento tenha sido lido pelo \textit{while} até à linha final.

Este cíclo é o motor de todo o nosso programa, que está a ser controlado através de flags e da função read, que ao ser executada, retorna o valor de bytes que foram lidos. Se este valor retornado for negativo, então o \textit{system call} está a retornar um erro. Esta é então a nossa maneira de parar o cíclo, executando-o até que o nosso read fique sem mais bytes para ler.

\lstinputlisting[breaklines=true, keywordstyle=\color{blue},frame=single,language=C, firstline=179, lastline=179]{notebook.c}
~\\

Antes de guardarmos numa variável, a linha lida é processada para que reconheça o conjunto de caratéres especiais para o correto processamento do ficheiro:

\begin{itemize}
\item \$; 
\item \$|;
\item \$(número)|;
\item \verb|>>>| e \verb|<<<|
\end{itemize}


\section{Sinais}

Uma das flags é usada para depuração de erros de qualquer fork criado. Se o fork retorna um \textit{status} diferente de sucesso, pode significar que a linha lida contém um comando errado.
Neste caso, teriamos então de cancelar a execução do nosso \textit{notebook}.


\lstinputlisting[breaklines=true, keywordstyle=\color{blue},frame=single,language=C, firstline=76, lastline=76]{notebook.c}
~\\

No entanto também pode haver ação humana e, para isso, o sinal \textit{SIGINT} é enviado para o programa (sinal normalmente relacionado com a combinação de botões \textit{Control+C}).
Nesse caso, a nossa variável global \textit{running} irá determinar o estado de execução do nosso programa.

\lstinputlisting[breaklines=true, keywordstyle=\color{blue},frame=single,language=C, firstline=35, lastline=38]{notebook.c}
~\\

A detecção do sinal deve ser inicializada na main pela função de sistema signal.



Uma linha delimitada por qualquer outra expressão diferente dos itens em cima irá ser ignorada e não interpretada como comando.


\raggedbottom
\pagebreak
\section{Re-processamento}

A funcionalidade normal do nosso programa será sempre executar e inserir os nossos resultados entre \verb|>>>| e \verb|<<<|.
Haverá no entanto alturas em que faremos alterações ao nosso sistema (criamos um ficheiro novo, o \textit{word count} de determinado ficheiro é agora maior, o estado de X dispositivo foi alterado, entre outros). 

\lstinputlisting[breaklines=true, keywordstyle=\color{blue},frame=single,language=C, firstline=104, lastline=104]{notebook.c}
~\\

Nestes casos, necessitamos então de voltar a executar as linhas de comando do nosso \textit{notebook}.

% FICHEIRO OU PIPE
A função re-processamento abre então um ficheiro temporário para podermos fazer o \textit{parsing} do nosso ficheiro de entrada, fazendo com que o original não seja imediatamente alterado.

\lstinputlisting[breaklines=true, keywordstyle=\color{blue},frame=single,language=C, firstline=111, lastline=112]{notebook.c}
~\\

Executamos então outro ciclo da mesma natureza do \textit{while} da função main, lendo linha a linha e, desta vez, ignorando todo o conteudo entre \verb|>>>| e \verb|<<<| do ficheiro de entrada.

Após isto tudo, usámos a função rename para relocarmos o ficheiro temporário para o original.

\lstinputlisting[breaklines=true, keywordstyle=\color{blue},frame=single,language=C, firstline=134, lastline=134]{notebook.c}
~\\





\section{Detecção de erros / Interrupção}

% CAP 2 - - - - - - - -  - - - - 

\chapter{Outras Funcionalidades}
\section{Histórico de comandos anteriores}
\section{Execução de conjuntos de comandos}

% CAP 3 - - - - - - - -  - - - - 


\chapter{Conclusão}
\end{document}